\section{Implementation} \label{section:implementation}

hDHT was implemented as a self contained library, as well as a daemon to act as the server.
The hDHT library also includes an example command line client, which allows one to test the capabilities of the system without physically moving around.

hDHT is open source, and available on Github~\footnote{\url{https://github.com/gcampax/hDHT}}. We implemented hDHT in approximately 5,000 lines of C++ code.

\subsection{Library Design}

hDHT uses C++, and the libuv library~\footnote{\url{http://libuv.org}} for the event loop and asynchronous portable IO.
It has been tested on Linux and OS X.

The client API of the library consists of a single \texttt{Client} object, which exposes methods to get and set the current location, get and set the local metadata, and perform queries against remote clients.
All other objects in the library, including the RTree implementation and the DHT abstraction, are hidden from library users.

To aid in code reuse, and reduce disk footprint, the library also includes the full server implementation. The server binary is a shim that initializes the library in server mode.

\subsection{R-Tree}
We built a self-contained R-Tree library from scratch by using the algorithms provided in the Hilbert R-Tree paper~\cite{kamel1993hilbert}. The \texttt{RTree} class provides a minimal interface to allow for inserting data
into the tree at a particular \texttt{Point} in the 2D space and performing a search using a \texttt{Rectangle}. The search returns a list of \texttt{NodeID}s that are located within the query rectangle.

\subsection{Peer Discovery and Connectivity Requirements}

In the current implementation, all peers are expected to be reachable by a direct TCP connection at least in one direction.

Peers are specified by address (IP and port) on the command line of the server.
If no peer is specified when the server starts, the server assumes control of the whole table, and waits for incoming connections.
All servers also listen for incoming connections on a specified IP and port, and publish this address to other servers.

\subsection{RPC Library}

To keep the library small, and avoid dependencies (especially dependencies that the authors were not familiar with when the project was started), hDHT uses a small home-grown RPC library, over TCP sockets.

The library uses a pseudo-IDL defined using the C++ preprocessor, which makes it avoid a separate compilation step, and makes it easy to integrate with IDEs and build systems.
It then makes use of C++ template metaprogramming to generate safe and efficient marshalling code.

All RPC requests in the library are asynchronous; the caller passes a callback (an \texttt{std::function}) to be notified of the completion of the RPC.
